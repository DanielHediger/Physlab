
%----------------------------------------------------------------------------------------
%	PACKAGES AND DOCUMENT CONFIGURATIONS
%----------------------------------------------------------------------------------------

\documentclass{article}

\usepackage[version=3]{mhchem} % Package for chemical equation typesetting
\usepackage{siunitx} % Provides the \SI{}{} and \si{} command for typesetting SI units
\usepackage{graphicx} % Required for the inclusion of images
\usepackage{natbib} % Required to change bibliography style to APA
\usepackage{amsmath} % Required for some math elements 
\usepackage{german}
\usepackage[utf8]{inputenc}
\setlength\parindent{0pt} % Removes all indentation from paragraphs
\renewcommand{\labelenumi}{\alph{enumi}.} % Make numbering in the enumerate environment by letter rather than number (e.g. section 6)

%\usepackage{times} % Uncomment to use the Times New Roman font

%----------------------------------------------------------------------------------------
%	DOCUMENT INFORMATION
%----------------------------------------------------------------------------------------

\title{Physiklabor \\ Laborbericht \\ Drehmoment und Drall} % Title

\author{Daniel \textsc{Hediger} \\ Lucien \textsc{Egloff}} % Author name



\date{\today} % Date for the report

\begin{document}

\maketitle % Insert the title, author and date

\begin{center}
\begin{tabular}{l r}
Ausführungsdatum: & September 28, 2016 \\ % Date the experiment was performed
Dozent: & Dr.Ackermann % Instructor/supervisor
\end{tabular}
\end{center}
\newpage
\tableofcontents 

%----------------------------------------------------------------------------------------
%	SECTION 1
%----------------------------------------------------------------------------------------
\newpage
\section{Aufgabe 1}


Das Trägheitsmoment eines Rades soll durch das anhängen eines Gewichtes bestimmt werden.

\subsection{Grundlagen}

Das Gewicht wird durch die Erdbeschleunigung nach unten gezogen, und durch das Seil wird das Rad beschleunigt. Durch diese Beziehung kann die Winkelbeschleunigung und die Trägheit berechnet werden.

\subsection{Versuchsaufbau}
\begin{figure}[h]
\center

\includegraphics[scale=0.3]{Wheel.pdf} 
\caption{Fahrradrad mit angehängten Gewicht.}
\
\end{figure}

\subsection{Berechnungen}
Energieerhaltung des Systems.\\\\
$E_{rot}+E_{kin}=E_{pot}  \Rightarrow\frac{1}{2}*I*\omega^2+\frac{1}{2}*m*v^2=m*g*h$\\\\
Das ganze nach dem Trägheitsmoment umgestellt.\\\\
$I=\frac{-m*(v^2-2g*h)}{\omega^2} $\\\\
Das fehlende $\omega$ wurde nun Experimentell bestimmt.

\subsection{Gemessene Grössen}
\begin{table}[h]
    \begin{tabular}{|l|l|l|l|}
        \hline
  
        Durchgang & RPM$\cdot$2 $ \quad $[$\frac{1}{60s}$]   & $\omega \quad $[$\frac{rad}{s}$] & I $\quad [kg\cdot m^2]$ \\ \hline
        1         & 114.3 & 11.97 & 0.076 \\ 
        2         & 131   & 13.72 & 0.055 \\ 
        3         & 118   & 12.36 & 0.071 \\ 
        4         & 117   & 12.25 & 0.072 \\ 
        5         & 121   & 12.67 & 0.066 \\ 
        6         & 116   & 12.15 & 0.073 \\ 
        7         & 117.6 & 12.31 & 0.071 \\ 
        8         & 126.2 & 13.22 & 0.060 \\ 
        9         & 119   & 12.46 & 0.069 \\
        \hline
    \end{tabular}
\end{table}
\section{Aufgabe 2}
Das eigene Trägheitsmoment soll Experimentell aufgrund der Drehimpulserhaltung bestimmt werden.     
\subsection{Grundlagen}
Durch die Gegebenheit der Drehzahl des Rades und dessen Trägheitsmoment, kann anhand der eigenen
Drehzahl nach dem verändern der Drehachse, das Trägheitsmoment des gesamten 
 Mensch-Hocker-Rad-Systems genähert werden.

\subsection{Versuchsaufbau}
\begin{figure}[h]
\center

\includegraphics[scale=0.5]{Drehschemmel.pdf} 
\caption{Person mit drehendem Fahrradrad}
\
\end{figure}
\subsection{Gemessene Grössen}
\subsection{Berechnungen}
Die Grundformel
$T_p = \frac{2*\pi*(I_m+I_s)*\omega}{(m_m+m_s)*g*h} $ wird nach $I_m$(I-Mensch)umgestellt.\\\\
$I_m = \frac{g*h*(m_m+m_s)*T_p}{2*\pi*\omega}-I_s$




\section{Aufgabe 3}
Ein rotierender Kreisel wird durch das anhängen eines Gewichtes zum Präzessieren gebracht werden. Daraus lässt sich das Trägheitsmoment berechnen. 
\subsection{Grundlagen}
\subsection{Gemessene Grössen}
\subsection{Berechnungen}
\subsection{Schlussfolgerung}

\section{Aufgabe 4}
Der Drallsatz soll durch das einwirken einer horizontalen Kraft bestimmt werden.
\subsection{Grundlagen}
Der Kreisel wird beschleunigt und anschliessend mit einer Horizontalen Kraft beeinflusst. Dadurch 
beginnt er sich zu drehen, und die Zeit die er für $90^\circ$ benötigt wird gemessen. 
Anschliessend werden wir mit Hilfe des Drallsatzes, der gemessenen Werte und des 
Trägheitsmoments die Kraft welche wir aufwendeten berechnen und auf Übereinstimmung 
überprüfen. 

\subsection{Berechnungen}
$F_{Zug}=\frac{I_{Kreisel}*2*\pi*n_{Kreisel}}{T_{90^\circ }*d*60}$
\subsection{Schlussfolgerung}

\section{Aufgabe 5}

Die zeitliche Abnahme der Rotationsfrequenz des Rades und des Kreisels aufgrund von
Reibung soll bestimmt werden.
\subsection{Grundlagen}
\subsection{Versuchsaufbau}
\subsection{Gemessene Grössen}
\subsection{Schlussfolgerung}
\end{document}