
%----------------------------------------------------------------------------------------
%	PACKAGES AND DOCUMENT CONFIGURATIONS BY DANIEL HEDIGER
%----------------------------------------------------------------------------------------

\documentclass{article}

\usepackage[version=3]{mhchem} % Package for chemical equation typesetting
\usepackage{siunitx} % Provides the \SI{}{} and \si{} command for typesetting SI units
\usepackage{graphicx} % Required for the inclusion of images
\usepackage{natbib} % Required to change bibliography style to APA
\usepackage{amsmath} % Required for some math elements 
\usepackage{german}
\usepackage{float}
\restylefloat{figure}
\usepackage[utf8]{inputenc}
\setlength\parindent{0pt} % Removes all indentation from paragraphs
\renewcommand{\labelenumi}{\alph{enumi}.} % Make numbering in the enumerate environment by letter rather than number (e.g. section 6)

%\usepackage{times} % Uncomment to use the Times New Roman font

%----------------------------------------------------------------------------------------
%	DOCUMENT INFORMATION
%----------------------------------------------------------------------------------------

\title{Physiklabor \\ Laborbericht \\ Magnetismus} % Title

\author{Daniel \textsc{Hediger} \\ Lucien \textsc{Egloff}} % Author name



\date{\today} % Date for the report

\begin{document}

\maketitle % Insert the title, author and date

\begin{center}
\begin{tabular}{l r}
Ausführungsdatum: & November 2, 2016 \\ % Date the experiment was performed
Dozent: & Dr.Ackermann \\% Instructor/supervisor
Version: & 1.0

\end{tabular}
\end{center}
\begin{figure}[H]
	\centering
	\includegraphics[scale=0.3]{Mag.pdf} 
\end{figure}
\newpage
\tableofcontents 

%----------------------------------------------------------------------------------------
%	SECTION 1
%----------------------------------------------------------------------------------------
\newpage
\section{Grundlagen}
Auf Elektronen in Magnetischen und elektrischen Feldern wirkt jeweils die Lorentzkraft. Mit Hilfe eines elektrischen Feldes können die Elektronen in einem Magnetfeld beschleunigt werden
. Bewegen sich die Elektronen zusätzlich in einem dünnen Gas, können sie sogar sichtbar gemacht werden.
In diesem Physiklabor haben wir daher die Lorentzkraft studiert und näher untersucht.  Das erste „Experiment“ war das kennenlernen der Hallsonde, welche benötigt wurde um das Magnetfeld zu messen.
Das zweite Experiment war das Messen des Magnetfeldes durch die Auslenkung einen Metallkonstruktes zu beobachten , beim dritten 
wurde das Magnetfeld in einer Helmholtzspule gemessen.
\section{Aufgabe Elektrisch durchgeflossener Leiter}
\subsection{Aufgabenstellung}
Die tangentiale und radiale Komponente des Magnetfeldes um einen stromführenden
geraden Leiter soll als Funktion des Abstands vom Leiter aufgezeigt werden.
\subsection{Versuchsaufbau}

\end{document}